%%%% IACR Transactions TEMPLATE %%%%
% This file shows how to use the iacrtrans class to write a paper.
% Written by Gaetan Leurent gaetan.leurent@inria.fr (2020)
% Public Domain (CC0)


%%%% 1. DOCUMENTCLASS %%%%
\documentclass[journal=tosc,notanonymus]{iacrtrans}
%%%% NOTES:
% - Change "journal=tosc" to "journal=tches" if needed
% - Change "submission" to "final" for final version
% - Add "spthm" for LNCS-like theorems


%%%% 2. PACKAGES %%%%
\usepackage{lipsum} % Example package -- can be removed


%%%% 3. AUTHOR, INSTITUTE %%%%
\author{Simon Hensel\inst{1} \and Helena Richter\inst{2}}
\institute{Albstadt-Sigmaringen University, Albstadt, Germany, \email{xxx@hs-albsig.de} \and
	Albstadt-Sigmaringen University, Albstadt, Germany, \email{richtehe@hs-albsig.de}}


%%%% NOTES:
% - We need a city name for indexation purpose, even if it is redundant
%   (eg: University of Atlantis, Atlantis, Atlantis)
% - \inst{} can be omitted if there is a single institute,
%   or exactly one institute per author


%%%% 4. TITLE %%%%
\title[Stateless Hash-Based Digital Signature Standard]{SLH-DSA: Stateless Hash-Based Digital Signature Standard}
\subtitle{Dokumentation zur Implementierung und Anwendung}
%%%% NOTES:
% - If the title is too long, or includes special macro, please
%   provide a "running title" as optional argument: \title[Short]{Long}
% - You can provide an optional subtitle with \subtitle.

\begin{document}

\maketitle


%%%% 5. KEYWORDS %%%%
\keywords{Something \and Something else}


%%%% 6. ABSTRACT %%%%
\begin{abstract}
  In this paper we prove that the One-Time-Pad has perfect security.

  \lipsum[8]
\end{abstract}


%%%% 7. PAPER CONTENT %%%%
\section{Introduction}

Widely used primitives like the AES~\cite{AES} do not have perfect
security, and can be analysed with linear
cryptanalysis~\cite{EC:Matsui93}, differential
cryptanalysis~\cite{JC:BihSha91}, or differential power
analysis~\cite{C:KocJafJun99}.  We show that the One-Time-Pad is
unconditionally secure in \autoref{sec:main}.

\lipsum[9]

\section{Exploring the SLH-DSA Signature Scheme}
\label{sec:overview}

\section{Implementation}
\label{sec:main}
\subsection{Functions and Addressing}

\lipsum[10]
\subsection{WOTS+}
\subsection{XMSS}
\subsection{FORS}


%%%% 8. BILBIOGRAPHY %%%%
\bibliographystyle{alpha}
\bibliography{abbrev3,crypto,biblio}
%%%% NOTES
% - Download abbrev3.bib and crypto.bib from https://cryptobib.di.ens.fr/
% - Use bilbio.bib for additional references not in the cryptobib database.
%   If possible, take them from DBLP.

\end{document}
