%%%% IACR Transactions TEMPLATE %%%%
% This file shows how to use the iacrtrans class to write a paper.
% Written by Gaetan Leurent gaetan.leurent@inria.fr (2020)
% Public Domain (CC0)


%%%% 1. DOCUMENTCLASS %%%%
\documentclass[journal=tosc,notanonymous]{iacrtrans}
%%%% NOTES:
% - Change "journal=tosc" to "journal=tches" if needed
% - Change "submission" to "final" for final version
% - Add "spthm" for LNCS-like theorems


%%%% 2. PACKAGES %%%%
\usepackage{lipsum} % Example package -- can be removed


%%%% 3. AUTHOR, INSTITUTE %%%%

\author{Simon Hensel\inst{1} \and Helena Richter\inst{2}}
\institute{Albstadt-Sigmaringen University, Albstadt, Germany, \email{hensels1@hs-albsig.de} \and
	Albstadt-Sigmaringen University, Albstadt, Germany, \email{richtehe@hs-albsig.de}}
%%%% NOTES:
% - We need a city name for indexation purpose, even if it is redundant
%   (eg: University of Atlantis, Atlantis, Atlantis)
% - \inst{} can be omitted if there is a single institute,
%   or exactly one institute per author


%%%% 4. TITLE %%%%
\title{Stateless Hash-Based Digital Signature Standard}
\subtitle{Dokumentation}
%%%% NOTES:
% - If the title is too long, or includes special macro, please
%   provide a "running title" as optional argument: \title[Short]{Long}
% - You can provide an optional subtitle with \subtitle.

\begin{document}

\maketitle


%%%% 5. KEYWORDS %%%%
\keywords{Something \and Something else}


%%%% 6. ABSTRACT %%%%
\begin{abstract}
  \lipsum[1]
\end{abstract}


%%%% 7. PAPER CONTENT %%%%
\section{Introduction}

In the realm of modern cryptography, ensuring data integrity and authenticity is a cornerstone of secure communication and digital transactions. Digital Signature Algorithms (DSAs) are pivotal in this context, enabling entities to validate the provenance and integrity of data without the need for direct interaction. Among the diverse array of DSAs, the Stateless Hash-based Digital Signature Algorithm (Stateless Hash-DSA) emerges as a robust and efficient cryptographic solution tailored for secure, lightweight, and future-proof applications.
\\
This project aims to explore and implement the Stateless Hash-DSA, an advanced algorithm designed to leverage the inherent strength of hash functions while addressing challenges associated with stateful signature schemes. Unlike traditional approaches that rely on maintaining and managing state information, Stateless Hash-DSA eliminates the complexities of state management, thereby reducing operational risks such as accidental state reuse—a common vulnerability in stateful systems.
\\
By implementing the Stateless Hash-DSA, this project contributes to the advancement of secure digital signature techniques suitable for environments where simplicity, efficiency, and resistance to quantum adversaries are crucial. The following documentation outlines the theoretical foundations, implementation details, and practical applications of the Stateless Hash-DSA.

\section{Hash-based Cryptography}

Stateless hash algorithms are cryptographic methods that leverage hash functions to ensure security while eliminating the need to maintain state information during operation. In traditional stateful algorithms, maintaining a record of past operations is essential to prevent vulnerabilities such as key or signature reuse. However, this reliance on state can introduce operational complexities and risks, particularly in distributed or constrained environments.
\\
By contrast, stateless hash algorithms operate without requiring a persistent state, relying solely on the cryptographic strength of hash functions to secure data. This design simplifies implementation, reduces the risk of errors associated with state management, and enhances resilience against attacks that exploit state inconsistencies. Stateless approaches are especially valuable in applications like digital signatures, where lightweight, efficient, and secure operations are critical.
\\
Stateless hash algorithms are also well-suited for post-quantum cryptography, leveraging the inherent robustness of hash functions against quantum adversaries, ensuring their applicability in future cryptographic systems.

\section{Digital Signature Algorithm (DSA)}
The Stateless Hash-based Digital Signature Algorithm (SLH-DSA) is a cryptographic scheme that combines the strength of hash-based security with the simplicity and efficiency of a stateless design. SLH-DSA is a robust alternative to traditional digital signature algorithms, addressing challenges such as state management, accidental key reuse, and scalability in distributed or resource-constrained environments.
\\
At its core, SLH-DSA employs cryptographic hash functions to generate secure, verifiable signatures. Unlike stateful algorithms that require persistent tracking of used keys or states to ensure security, SLH-DSA eliminates this dependency by deriving keys and signatures dynamically in a manner that guarantees their uniqueness and integrity. This stateless approach significantly reduces the operational risks and complexities associated with managing and safeguarding state information.\\
\\
The Algorithm can be divided into 4 Steps:
\begin{enumerate}
	\item \textbf{Key Generation} of private and public key
	\item \textbf{Key Distribution}  of public key
	\item \textbf{Signature Generation} by sender
	\item \textbf{Signature Verification} by receiver
\end{enumerate}
During the \textbf{Key Generation} both a private key and a public key are created. The private key consists of two main components:
the \texttt{SK.seed} which is a random seed used to generate all the secret values for the WOTS+ (Winternitz One-Time Signature Plus) and FORS (Forest of Random Subsets) keys and the \texttt{SK.prf}, a pseudorandom function (PRF) key used to generate a randomization value for the message hashing during signature generation.
Additionally, the private key includes a copy of the public key components (PK.seed and PK.root) for use during signature generation and verification.
\\
The public key consists of the \texttt{PK.seed} which is a public seed used to provide domain separation between different SLH-DSA key pairs. It is also used in many hash function calls. The key also konsists of the\texttt{PK.root} which is the root of the top-layer XMSS (eXtended Merkle Signature Scheme) tree, which serves as the public key of the hypertree.
\\
The key generation algorithm (\verb|slh_keygen_internal|) takes \texttt{SK.seed}, \texttt{SK.prf}, and \texttt{PK.seed} as inputs.
It computes the root of the top-layer XMSS tree (\texttt{PK.root}) using the \verb|xmss/_node| function, which recursively constructs the Merkle tree from the WOTS++ public keys.
The private key is then formed by bundling \texttt{SK.seed}, \texttt{SK.prf}, \texttt{PK.seed}, and \texttt{PK.root}.
The public key is formed by bundling \texttt{PK.seed} and \texttt{PK.root}.
\\
Both \texttt{SK.seed} and \texttt{SK.prf} must be generated using an approved random bit generator (RBG) with a security strength of at least 8\texttt{n} bits, where \texttt{n} is the security parameter (16, 24, or 32 bytes depending on the parameter set).\\
\newline
\textbf{Key Distribution} in SLH-DSA follows standard cryptographic practices, with a focus on securely sharing public keys while keeping private keys confidential. This ensures that the system remains secure and resistant to attacks, even in a post-quantum setting. Since the public key is not sensitive, it can be distributed freely. Common methods for distributing public keys include Public Key Infrastructure (the public key can be shared via digital certificates issued by a trusted Certificate Authority (CA)), Direct Sharing or Key Servers.\\
\newline
The \textbf{signing} process begins with the message (\texttt{M}) that needs to be signed, along with the signer's private key (\texttt{SK}). 
To ensure that each signature is unique, even if the same message is signed multiple times, a randomizer (\texttt{R}) is generated. This is done using the \texttt{PRFmsg} function, which takes \texttt{SK.prf} and the message \texttt{M} as inputs. The randomizer is then combined with the message, \texttt{PK.seed}, and \texttt{PK.root} using the \texttt{} function to produce a message digest. This digest serves as the basis for the rest of the signing process.\\
The message digest is divided into parts to determine how the signature will be constructed. One portion of the digest is used to select a FORS key from the large set of FORS key pairs in the SLH-DSA key structure. Another portion is used to determine the index of the XMSS tree and the specific WOTS+ key within that tree. This ensures that the signature is tied to a specific FORS key and XMSS tree.\\
Once the FORS key is selected, it is used to sign part of the message digest. This involves generating a FORS signature using the \verb|fors_sign| function. The FORS signature includes secret values from the FORS private key, along with their corresponding authentication paths in the FORS Merkle trees. These authentication paths allow the verifier to reconstruct the FORS public key during verification.\\
After the FORS signature is created, the corresponding FORS public key is computed. This public key is then signed using the hypertree structure, which consists of multiple layers of XMSS trees. The \verb|ht_sign| function generates a hypertree signature by creating a sequence of XMSS signatures. The process starts at the bottom layer of the hypertree, where the FORS public key is signed, and moves up through each layer until the top layer is reached. Each layer signs the public key of the layer below it, ultimately producing a chain of signatures that authenticate the FORS public key.\\
The final SLH-DSA signature is composed of three main components:
the randomizer (\texttt{R}), which ensures the uniqueness of the signature, the FORS signature, which signs part of the message digest and the hypertree signature, which authenticates the FORS public key through the layered XMSS structure.\\
\newline
The \textbf{verification} process starts with three key inputs: the original message (\texttt{M}), the signature (\texttt{SIG}), and the signer's public key (\texttt{PK}). The signature consists of three components: the randomizer (R\texttt{text}), the FORS signature, and the hypertree signature. The public key includes \texttt{PK.seed} (a public seed) and \texttt{PK.root} (the root of the top-layer XMSS tree), which are essential for the verification process.\\
The verifier starts by recomputing the message digest using the same process that the signer used during the signing process. This involves hashing the message \texttt{M} together with the randomizer \texttt{R}, \texttt{PK.seed}, and \texttt{PK.root} using the \texttt{Hmsg} function. The resulting digest should match the one used by the signer to generate the signature. If the digest does not match, the signature is immediately invalid.\\
The recomputed message digest is split into parts to determine the specific keys and trees used in the signing process. One portion of the digest is used to identify the FORS key that was used to sign part of the message. Another portion determines the index of the XMSS tree within the hypertree, and a third portion identifies the specific WOTS++ key within that XMSS tree. These indices are important for reconstructing the FORS public key and verifying the hypertree signature.\\
Using the FORS signature and the relevant portion of the message digest, the verifier computes a candidate FORS public key using the \verb|fors_pkFromSig| function. This involves reconstructing the Merkle tree roots from the FORS signature and the message digest. The verifier then hashes these roots together to produce the candidate FORS public key. This step ensures that the FORS signature corresponds to the correct FORS key and that the message digest was signed properly.\\
The candidate FORS public key is then verified using the hypertree signature. The hypertree signature consists of a sequence of XMSS signatures, starting from the bottom layer of the hypertree (which signs the FORS public key) and moving up to the top layer (which signs the XMSS public key of the layer below). The verifier uses the \verb|ht_verify| function to check each XMSS signature in the sequence. For each layer, the verifier reconstructs the corresponding XMSS public key using the \verb|xmss_pkFromSig| function and compares it to the expected value. This process continues until the top layer is reached.\\
At the top layer of the hypertree, the verifier compares the computed XMSS public key to the \texttt{PK.root} from the signer's public key. If the two values match, the signature is considered valid. This confirms that the signature was generated using the signer's private key and that the message has not been altered. If the computed public key does not match \texttt{PK.root}, the signature is invalid, indicating either tampering with the message or an incorrect signature. \cite{national-institute-of-standards-and-technology-2024}\\
\newline
SLH-DSA is especially useful for modern applications where lightweight and scalable solutions are critical. It is resilient to common vulnerabilities found in stateful systems and offers enhanced security against evolving threats, including those posed by quantum computing. Furthermore, its reliance on well-studied cryptographic hash functions ensures that it remains a practical and secure choice for both current and future cryptographic needs.
\\
This algorithm is a key step toward creating secure, efficient, and future-proof digital systems, making it a compelling choice for developers and security professionals seeking innovative cryptographic solutions.


\section{Implementing the Algorithm}

First, we implement a prototype of SLH-DSA in Python.
This has numerous advantages: First, Python code is rather easy to run and debug since no compilation is needed.
Second, the whole signature algorithm can be implemented using only the Python standard library.
After the Python protoype is confirmed to be working, we implement the algorithm again in the programming language C, which offers vastly faster execution times compared to Python.
With this approach, we can always make sure the C implementation is working by comparing it with the Python protoype.
For both implementations, we choose to use one dedicated file per chapter in the standard specification, so the code mimicks the overall structure of the specification paper.
% FIXME maybe we cant cite the NIST github repo properly instead of only providing a footnote with a link
Both implementations are tested with the official test vectors provided by the NIST on their official GitHub page \cite{usnistgov-2024}.


\subsection{Python}

The Python implementation is rather straightforward.
We started with chapter 4 of the specification, which governs the pseudo-random functions and hash addresses.
All functions regarding the hash address are grouped up in a class, all other functions use regular methods.
The parameter sets are provided via a global variable that gets initialized at the beginning of the program.
Altough global variables are considered bad practice in general, we deliberately take this approach as now the parameters do not have to be passed to every function call individually.
In addition, the parameters never change during runtime, so we do not have to take race conditions or similar issues into account.
\\
The Python implemented always operates on \texttt{byte}-objects, the initial message, as well as the key values and the hash outputs are bytes.
All pseudo-random functions are implemented using Python's standard module \texttt{hashlib} to generate the various hashes.
Furthermore, the \texttt{secrets} module is used to generate cryptographically secure random values during the key generation and for the non-deterministic variant of the algorithm.
\\
The rest of the algorithm pretty much follows the pseudocode provided in the specification.
In cases where floor division is decessary, we use Python's floor operator \texttt{//}, if the ceiling of a value has to be calculated, we add the $denominator - 1$ to the numerator.
For instance, $\lceil \frac{k * a}{8} \rceil$ becomes \texttt{(k * a + 7) // 8}.


\subsection{C}

The C code now gets implemented after the Python protoype is finished and confirmed to be working.
We take the same approach of using one code file per chapter, which in the context of C means one \texttt{.c} file accompanied by a \texttt{.h} file with the same name.
Here, all parameters are stored in a struct, that gets initialized at the beginning of the program and whose address passed to each function call individually.
Similarly, the address is now also a struct, since C does not support classes.
\\
The code operates on byte arrays, the message that gets signed, as well as the keys, the hash outputs and the signatures are stored in byte arrays.
As such, every function that returns a byte value, has an additional input called \texttt{buffer}, which is a pointer to an array where the output values get copied to.
Great care has been taken to avoid any form of buffer overflows or overreads when doing so.
Furthermore, we do not use the \texttt{malloc()} to avoid problems such as memory leaks when forgetting to call \texttt{free()} or segmentation faults when trying to access memory that has already been free'd.
\\
Ceiling and floor division is implemented in the same way as the Python protoype, since the math library of C returns values of data type \texttt{double} when calling \texttt{ceil()} or \texttt{floor()}, which need to be converted back to an integer type.
While this is generally possible, the larger parameter sets, namely \texttt{SLH-DSA-SHAKE-192s} and \texttt{SLH-DSA-SHAKE-256s}, produce values that are 64 bit in size, which causes floating point exeptions when trying to cast from a double to a 64 bit integer.
\\
The pseudo-random hash functions are first implemented using the \texttt{libgcrypt} library, which is available on many Linux platforms.
In a second iteration, we replace \texttt{libgcrypt} with the \texttt{kcp/optimized1600AVX512} implementation, which is part of the supercop cryptography benchmark system, for the pseudo-random functions.
Random numbers are generated using the cross-platform \texttt{sodium} library.
\\
While not impacting performance, we also take measures to modernize the C codebase.
For instance, we use the datatypes defined in \texttt{stdint.h}, so \texttt{unsigned char} becomes \texttt{uint8\_t}, \texttt{unsigned int} becomes \texttt{uint32\_t} and so on.
In addition, we replace header include guards with the simple macro \texttt{\#pragma once}, which achives the same effect without having to make our own definitions.


%%%% 8. BILBIOGRAPHY %%%%
\bibliographystyle{alpha}
\bibliography{biblio}
%%%% NOTES
% - Download abbrev3.bib and crypto.bib from https://cryptobib.di.ens.fr/
% - Use bilbio.bib for additional references not in the cryptobib database.
%   If possible, take them from DBLP.

\end{document}
